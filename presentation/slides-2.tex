\documentclass{beamer}

\usetheme{Szeged}

\usepackage[utf8]{inputenc}
\usepackage{bussproofs}
\usepackage[]{algorithm2e}

\author{Jannik Vierling \and Thomas Baumhauer}
\title{Stackbasierte Sprachen -- WS 2017}
\subtitle{Aussagenlogischer Resolutionsbeweiser}

\begin{document}

\frame{
  \maketitle
}

\frame{
  \frametitle{Aussagenlogisches Resolutionsbeweisen}
  Resolution:
  \begin{prooftree}
    \AxiomC{$\Gamma \rightarrow \Delta, A$}
    \AxiomC{$A, \Pi \rightarrow \Lambda$}
    \BinaryInfC{$\Gamma, \Pi \rightarrow \Delta, \Lambda$}
  \end{prooftree}
  Resolutionsbeweise:
  \begin{itemize}
  \item Resolution auf Klauseln $S_0$ anwenden bis \dots
    \begin{itemize}
    \item Leerklausel $\rightarrow$ abgeleitet {\color{blue}$\leadsto$ UNSAT} 
    \item keine neue Klausel ableitbar {\color{blue}$\leadsto$ SAT}
    \end{itemize}
    \item Beweis für Unerfüllbarkeit von $S_0$ (bzw. Gültigkeit von $\neg S_0$).
  \end{itemize}
}

\frame{
	\frametitle{{[ -2 ][ 1 2 ][ -1 2 ]} refute}
	\texttt{\scriptsize
	Seen:\hspace{1.5em} { }					 \\
	Working: { [ -2 ][ 1 2 ][ -1 2 ]}					 \\
	-----------------------					 \\
	Seen:\hspace{1.5em} { [ -2 ]}					 \\
	Working: { [ -1 2 ][ 1 2 ]}					 \\
	-----------------------					 \\
	Seen:\hspace{1.5em} { [ -2 ][ -1 2 ]}					 \\
	Working: { [ -1 ][ 1 2 ]}					 \\
	-----------------------					 \\
	Seen:\hspace{1.5em} { [ -2 ][ -1 2 ][ -1 ]}					 \\
	Working: { [ 1 2 ]}					 \\
	-----------------------					 \\
	Seen:\hspace{1.5em} { [ -2 ][ -1 2 ][ 1 2 ][ -1 ]}					 \\
	Working: { [ 1 ][ 2 ]}					 \\
	-----------------------					 \\
	Seen:\hspace{1.5em} { [ -2 ][ -1 2 ][ 1 2 ][ -1 ][ 1 ]}					 \\
	Working: { [ 2 ][ ]}					 \\
	=======================					 \\
	UNSAT ok
	}
}

\frame{
	\frametitle{{[ 1 2 ][ -1 2 ][ 2 3 ]} refute}
	\texttt{\scriptsize
	Seen: { }					 \\
	Working: { [ 1 2 ][ -1 2 ][ 2 3 ]}					 \\
	-----------------------					 \\
	Seen:\hspace{1.5em}  { [ 1 2 ]}					 \\
	Working: { [ -1 2 ][ 2 3 ]}					 \\
	-----------------------					 \\
	Seen:\hspace{1.5em} { [ 1 2 ][ -1 2 ]}					 \\
	Working: { [ 2 ][ 2 3 ]}					 \\
	-----------------------					 \\
	Seen:\hspace{1.5em} { [ 1 2 ][ -1 2 ][ 2 ]}					 \\
	Working: { [ 2 3 ]}					 \\
	-----------------------					 \\
	Seen:\hspace{1.5em} { [ 1 2 ][ -1 2 ][ 2 3 ][ 2 ]}					 \\
	Working: { }					 \\
	=======================					 \\
	SAT ok
	}
}

\frame{
	\frametitle{Programmiertechnik: ``Streams''}
	Funktion liefert keine feste Zahl von Rückgabewerten
	$\Rightarrow$ lege Lösungen und Zähler auf den Stack.
			
	$ $
	
	{\bf Beispiel:} Resolvieren zweier Klauseln
	
	$ $
	
	\texttt{\scriptsize
	[ 1 2 3 ] [ -1 -2 4 ] clause.resolve\_full  ok \\
	.s <3> 17240464 17240624 2  ok \\
	make-set-from-stack  ok \\
	.s <1> 17240368  ok \\
	show-set \{[ -2 2 3 4 ][ -1 1 3 4 ]\} ok \\
	}
}

\frame{
	\frametitle{Programmiertechnik: ``Streams''}
	
	Vergleiche in Haskell: Generator/Selektor-Prinzip
	
	$ $
	
	\texttt{\scriptsize
		fibs = 0 : 1 : zipWith (+) fibs (tail fibs) \\
		take 5 fibs \\
		{[0,1,1,2,3]}
	}
}

\end{document}
