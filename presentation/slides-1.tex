\documentclass{beamer}

\usetheme{Szeged}

\usepackage[utf8]{inputenc}
\usepackage{bussproofs}
\usepackage[]{algorithm2e}

\author{Jannik Vierling \and Thomas Baumhauer}
\title{Stackbasierte Sprachen -- WS 2017}
\subtitle{Aussagenlogischer Resolutionsbeweiser}

\begin{document}

\frame{
  \maketitle
}

\section{Resolutionsbeweisen}

\frame{
  \frametitle{Problemstellung}

  Automatisches Beweisen: Ist eine gegebene Formel gültig? Gibt es einen Beweis? \bigskip \\

  {\color{blue}Problem} UNSAT \\
  {\color{blue}Instanz}: Aussagenlogische Formel $\varphi$ in CNF. \\
  {\color{blue}Frage}: Ist die Formel $\varphi$ unerfüllbar? \\
  {\color{blue}Ausgabe}: Beweis für die unerfüllbarkeit \bigskip \\

  $\leadsto$ Komplement von SAT  $\Rightarrow$ UNSAT ist $\mathsf{co}\mathbf{NP}$-vollständig \\
  $\leadsto$ Nicht-effizient lösbar (?)
}

\frame{
  \frametitle{Das Resolutionsprinzip}
  \begin{center}
  CNF $\leftrightsquigarrow$ Menge atomarer Sequenten
  \end{center}
  {\small
  \begin{definition}[Atomarer Sequent]
  \[
    a_1, \dots, a_n \rightarrow b_1, \dots, b_m
  \]
  \end{definition}
  \begin{definition}[Resolution]
  \begin{prooftree}
    \AxiomC{$\Gamma \rightarrow \Delta, A$}
    \AxiomC{$A, \Pi \rightarrow \Lambda$}
    \BinaryInfC{$\Gamma, \Pi \rightarrow \Delta, \Lambda$}
  \end{prooftree}
\end{definition}
}
\begin{center}
Resolutionsbeweisen $\leftrightsquigarrow$ Erschöpfende Resolution
\end{center}
}

\frame{
  \frametitle{Ein Einfaches Beispiel}
  \begin{gather*}
    (A \vee B \vee \neg C) \wedge \neg B \wedge \neg A \wedge C \\ \\
    \{ C \rightarrow A, B \quad ; \quad B \rightarrow \quad ; \quad A \rightarrow \quad ; \quad \rightarrow C \}
  \end{gather*}
  \begin{prooftree}
    \AxiomC{$\rightarrow C$}

    \AxiomC{$C \rightarrow A, B$}
    \AxiomC{$B \rightarrow$}
    \BinaryInfC{$C \rightarrow A$}

    \AxiomC{$A \rightarrow$}
    
    \BinaryInfC{$C \rightarrow$}

    \BinaryInfC{$\phantom{C} \rightarrow$}
  \end{prooftree} \medskip
  {\color{blue}$\leadsto$} Die Formel ist unerfüllbar
}

\section{Implementierung}

\frame{
  \frametitle{Algorithmus}
  \begin{algorithm}[H]
    \KwData{Formel $\varphi$ in CNF}
 \KwResult{SAT oder UNSAT und Resolutionsdeduktion}
 $W = \varphi$\;
 $S = \varnothing$\;
 $N = \varnothing$\;
 \While{$\rightarrow \notin W$ and $W \neq \varnothing$}{
  wähle $C \in W$\;
  $W = W \setminus \{ C \}$\;
  $S = S \cup \{ C \}$\;
  $N = \mathrm{Res}(C, S) \setminus (W \cup S)$\;
  $W = W \cup N$\;
 }
\end{algorithm}
}

\frame{
  \frametitle{Besonderheiten}
  \begin{itemize}
    \item Übergabe variabler Anzahl von Argumenten.
    \item Konkatenativer Programmierstil.
    \item Maßgeschneiderte Syntax für Problemeingabe.
    \item Eigene Kontrollstrukturen.
    \item Generische Wörter.
  \end{itemize}
}

\end{document}